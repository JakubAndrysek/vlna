% Text from https://github.com/Pletacka-IoT
% \documentclass{template/socthesis}
\documentclass[12pt, a4paper]{article}

\usepackage[utf8]{inputenc}
\usepackage[czech,shorthands=off]{babel}

% \usepackage[T1]{fontenc} % výstupní kódování
\usepackage{subcaption}
\usepackage{amsmath}
\usepackage{enumitem}
\usepackage{hyperref} % reference
\usepackage{gensymb} % balíček symbolů
\usepackage{booktabs}

\usepackage[toc,page]{appendix}
\usepackage{color} % balíček pro obarvování textů
\usepackage{xcolor}  % zapne možnost používání barev, mj. pro \definecolor
\definecolor{mygreen}{RGB}{0,150,0} % nastavení barev odkazů
\usepackage{listings} % balíček pro formátování zdrojových kódů
\usepackage{multirow}
\usepackage{pifont}
\usepackage{pdfpages}
\usepackage{parskip}
\usepackage[margin=25mm]{geometry}
\usepackage[backend=bibtex]{biblatex}
\usepackage[author=,status=draft]{fixme} % vkládání poznámek
% dva módy (status): draft (poznámky se zobrazují v PDF) / final (poznámky se nezobrazují v PDF)



\newcommand{\cmark}{\textcolor{green}{\ding{51}}}%
\newcommand{\xmark}{\textcolor{red}{\ding{55}}}%

\lstset { %
    language=C++,
    backgroundcolor=\color{black!5}, % set backgroundcolor
    basicstyle=\footnotesize,% basic font setting
}

\setlist[itemize]{parsep=1pt}

\addbibresource{text.bib} % soubor s bibliografií
% \nocite{*}


% hinty k používání balíčků hyperref, url, hyperlink a hypertarget
% \usepackage{hyperref} % balíček pro hypertextové odkazy
% \url{www.odkaz.cz}
% \href{http://www.odkaz.cz}{Text který bude jako odkaz}
% \hyperlink{label}{proklikávací_text} - odkaz na text
% \hypertarget{label}{cíl_odkazu} - cíl odkazu

\begin{document} % konec preambule dokumentu


\input{titlepage.tex}
\newpage

% #########################################################################################
\section*{Abstrakt}
Tato práce představuje ucelený systém pro automatický monitoring průmyslové výroby a
    jeho nasazení do praxe.
Systém byl navrhován jako univerzální platforma pro monitoring výrobních strojů, s primárním zaměřením na pletací stroje.

Cílem této práce bylo navrhnout ucelený systém, který dokáže:

\begin{itemize}
    \item automaticky počítat upletené ponožky
    \item on-line hlásit poruchu na stroji a zjišťovat celkovou poruchovost strojů
    \item porovnávat výkonnost jednotlivých pracovních směn
    \item monitorovat průběh výroby
    \item nahradit část monotónní práce operátora
    \item zrychlit a zefektivnit výrobu
    \item snížit chybovost
\end{itemize}

Systém se skládá ze senzorové části, serveru a podpůrného serveru.
Senzorová část je postavená na vlastní senzorové desce s
mikrokontrolérem s barevným displejem. Senzor je připojen k pletacímu stroji a odesílá naměřená data.
Server přijímá  data ze senzorů, zajišťuje jejich zpracování a následné zobrazení uživateli.
Podpůrný server se stará o aktualizaci a o kontrolu správného chodu senzorů a
tady

asd 9 asd



\end{document}